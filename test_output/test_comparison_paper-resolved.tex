\documentclass[11pt]{article}

\usepackage{amsmath, amsthm, amssymb}
\usepackage{hyperref}
\usepackage{geometry}
\usepackage{booktabs}

\title{The Crystallographic Restriction Theorem:\\
A Formalized Proof in Lean 4}
\author{James Kuzmanovich\thanks{Wake Forest University. Original paper published in
\textit{The American Mathematical Monthly}, 109(2):173--186, 2002.}
\and
Andrey Pavlichenkov
\and
Eric Vergo\thanks{Formalization in Lean 4.}}
\date{}

\geometry{margin=1in}
\theoremstyle{plain}
\newtheorem{theorem}{Theorem}[section]
\newtheorem{lemma}[theorem]{Lemma}
\newtheorem{proposition}[theorem]{Proposition}
\newtheorem{corollary}[theorem]{Corollary}
\theoremstyle{definition}
\newtheorem{definition}[theorem]{Definition}
\newtheorem{example}[theorem]{Example}
\theoremstyle{remark}
\newtheorem{remark}[theorem]{Remark}
\newcommand{\Z}{\mathbb{Z}}
\newcommand{\Q}{\mathbb{Q}}
\newcommand{\N}{\mathbb{N}}
\newcommand{\C}{\mathbb{C}}
\newcommand{\GL}{\mathrm{GL}}
\newcommand{\ord}{\mathrm{ord}}
\newcommand{\lcm}{\mathrm{lcm}}
\newcommand{\diag}{\mathrm{diag}}

\begin{document}

\maketitle

\begin{abstract}...\end{abstract}\subsection{Introduction}

Finite groups of matrices with integer entries arise naturally in many areas of mathematics and physics. In crystallography, the symmetry groups of crystal lattices are realized as finite subgroups of $\GL(n, \Z)$, the group of $n \times n$ matrices with integer entries whose inverses also have integer entries. A fundamental question is: \emph{which finite orders can elements of $\GL(n, \Z)$ have?}

The answer involves a beautiful interplay between linear algebra, number theory, and algebra. Minkowski proved the remarkable result that $\GL(n, \Z)$ contains only finitely many isomorphism classes of finite subgroups, which implies that there are only finitely many possible orders for elements of finite order. The crystallographic restriction theorem provides a precise characterization of these orders.

\subsubsection{Historical Context}

The name ``crystallographic restriction'' originates from the physical study of crystals. A crystal lattice in $\mathbb{R}^n$ is a discrete subgroup of translations, and the rotational symmetries of such a lattice must preserve the lattice structure. This requirement severely constrains the possible rotation angles.

In two and three dimensions, the classical crystallographic restriction states that a rotation preserving a lattice must have order 1, 2, 3, 4, or 6. This explains why crystals can have 2-fold, 3-fold, 4-fold, or 6-fold rotational symmetry, but never 5-fold or 7-fold symmetry---a fact observed experimentally long before it was proved mathematically.

The general theorem we formalize extends this to all dimensions, answering completely which orders are achievable in $\GL(N, \Z)$ for any $N$.

\subsubsection{Statement of the Main Result}

The characterization involves a function $\psi : \N \to \N$ that we call the \emph{dimensional cost function}. Informally, $\psi(m)$ measures the minimum dimension needed to construct an integer matrix of order $m$.

\begin{theorem}
\label{thm:main-intro}
Let $N$ be a positive integer. An $N \times N$ matrix with integer entries can have finite order $m$ if and only if $\psi(m) \leq N$. 

\end{theorem}

The function $\psi$ is defined as follows. For a prime power $p^k$: 

\[

\psi_{\mathrm{pp}}(p, k) = \begin{cases}
0 & \text{if } k = 0, \\
0 & \text{if } p = 2 \text{ and } k = 1, \\
\varphi(p^k) & \text{otherwise,}
\end{cases}

\]

where $\varphi$ denotes Euler's totient function. For a general positive integer $m = \prod_i p_i^{k_i}$, we define $\psi(m) = \sum_i \psi_{\mathrm{pp}}(p_i, k_i)$.

The special treatment of $2^1$ reflects the fact that order 2 is ``free'' in any positive dimension: the matrix $-I$ has order 2 in any dimension $N \geq 1$.

\subsubsection{Significance and Applications}

The crystallographic restriction has several notable consequences:

\begin{enumerate}
\item \textbf{Classical 2D/3D restriction:} Since $\psi(m) \leq 2$ only for $m \in \{1, 2, 3, 4, 6\}$, these are the only achievable orders in dimensions 2 and 3, explaining the observed rotational symmetries of crystals.

\item \textbf{Dimension parity:} Since $\psi(m)$ is always even for $m > 2$, the achievable orders in dimension $2k$ are the same as in dimension $2k+1$. This surprising result means that odd dimensions offer no new orders beyond their even predecessors.

\item \textbf{First occurrence of orders:} The theorem determines exactly when each order first becomes achievable. For instance, order 5 requires dimension at least 4 (since $\psi(5) = \varphi(5) = 4$), while order 7 requires dimension at least 6. 

\end{enumerate}

\subsubsection{Overview of the Proof}

Our formalization follows the proof structure of Kuzmanovich and Pavlichenkov~\cite{kuzmanovich2002}. The proof proceeds in two directions:

If an $N \times N$ integer matrix $A$ has order $m$, we show $\psi(m) \leq N$. The key insight is that the minimal polynomial of $A$ (viewed over $\Q$) must be a product of distinct cyclotomic polynomials $\Phi_d$ for various divisors $d$ of $m$. The constraint that the order is exactly $m$ forces the lcm of these divisors to equal $m$, leading to the dimension bound.

If $\psi(m) \leq N$, we construct an explicit $N \times N$ integer matrix with order $m$. The construction uses companion matrices of cyclotomic polynomials combined via block diagonal matrices.

The formalization required developing substantial supporting infrastructure, including properties of cyclotomic polynomials, companion matrices, block diagonal constructions, and the $\psi$ function itself.

\subsection{Preliminaries}

We establish notation and recall the key definitions needed for the proof.

\subsubsection{Notation and Conventions}

Throughout, we use the following notation: 

\begin{itemize}
\item $\N$ denotes the natural numbers $\{0, 1, 2, \ldots\}$ 

\item $\Z$ denotes the integers 

\item $\Q$ denotes the rational numbers 

\item $\GL(N, \Z)$ denotes the group of $N \times N$ invertible integer matrices 

\item $\varphi(m)$ denotes Euler's totient function 

\item $\Phi_m(X)$ denotes the $m$-th cyclotomic polynomial 

\item $\ord(A)$ denotes the multiplicative order of a matrix $A$ (the smallest positive $k$ with $A^k = I$) 

\end{itemize}

For a matrix $A$ with $A^m = I$ for some $m > 0$, we say $A$ has \emph{finite order}, and $\ord(A)$ is the smallest such $m$.

We write $p^k \| m$ to mean that $p^k$ is the exact power of $p$ dividing $m$, that is, $p^k \mid m$ but $p^{k+1} \nmid m$.

\subsubsection{The Psi Function}

The dimensional cost function $\psi$ is central to the crystallographic restriction. We first define it on prime powers.

\begin{definition}
\label{def:psi-pp}
For a prime $p$ and exponent $k \geq 0$, define 

\[

\psi_{\mathrm{pp}}(p, k) = \begin{cases}
0 & \text{if } k = 0, \\
0 & \text{if } p = 2 \text{ and } k = 1, \\
\varphi(p^k) & \text{otherwise.}
\end{cases}

\]

\end{definition}

The function $\psi_{\mathrm{pp}}(p, k)$ equals $\varphi(p^k) = (p-1)p^{k-1}$ in most cases, with two exceptions: $\psi_{\mathrm{pp}}(p, 0) = 0$ for any prime $p$, and $\psi_{\mathrm{pp}}(2, 1) = 0$. The latter exception captures the fact that order 2 requires no ``dimensional cost'' since $-I$ has order 2 in any positive dimension.

\begin{definition}
\label{def:psi}
For a positive integer $m$ with prime factorization $m = \prod_i p_i^{k_i}$, define 

\[

\psi(m) = \sum_i \psi_{\mathrm{pp}}(p_i, k_i).

\]

Equivalently, $\psi(m) = \sum_{\substack{p^k \| m \\ (p,k) \neq (2,1)}} \varphi(p^k)$. 

\end{definition}

\subsubsection{Properties of Psi}

Several properties of $\psi$ are essential for the proof.

\begin{lemma}
\label{lem:psi-values}
For small values of $m$: 

\begin{center}...\end{center}\end{lemma}

\begin{proof}
Direct computation: $\psi(1) = 0$, $\psi(2) = 0$ (special case), $\psi(3) = \varphi(3) = 2$, $\psi(4) = \varphi(4) = 2$, $\psi(5) = \varphi(5) = 4$, $\psi(6) = \psi(2) + \psi(3) = 0 + 2 = 2$, $\psi(7) = \varphi(7) = 6$, $\psi(8) = \varphi(8) = 4$, $\psi(9) = \varphi(9) = 6$, $\psi(10) = \psi(2) + \psi(5) = 0 + 4 = 4$, $\psi(11) = \varphi(11) = 10$, $\psi(12) = \psi(4) + \psi(3) = 2 + 2 = 4$. 

\end{proof}

\begin{lemma}
\label{lem:psi-coprime}
If $\gcd(m, n) = 1$, then $\psi(mn) = \psi(m) + \psi(n)$. 

\end{lemma}

\begin{proof}
Since $m$ and $n$ are coprime, their prime factorizations are disjoint: if $m = \prod_i p_i^{a_i}$ and $n = \prod_j q_j^{b_j}$, then $mn = \prod_i p_i^{a_i} \cdot \prod_j q_j^{b_j}$ with all primes $p_i, q_j$ distinct. By definition, 

\[

\psi(mn) = \sum_i \psi_{\mathrm{pp}}(p_i, a_i) + \sum_j \psi_{\mathrm{pp}}(q_j, b_j)
= \psi(m) + \psi(n). \qedhere

\]

\end{proof}

\begin{lemma}
\label{lem:psi-totient}
For all $m \geq 1$, we have $\psi(m) \leq \varphi(m)$. 

\end{lemma}

\begin{proof}
It suffices to show $\psi_{\mathrm{pp}}(p, k) \leq \varphi(p^k)$ for each prime power in the factorization of $m$, then sum. For $k = 0$, both sides are 0. For $k \geq 1$ with $(p, k) \neq (2, 1)$, we have $\psi_{\mathrm{pp}}(p, k) = \varphi(p^k)$. For $(p, k) = (2, 1)$, $\psi_{\mathrm{pp}}(2, 1) = 0 < 1 = \varphi(2)$. 

\end{proof}

\begin{remark}
The only case where $\psi(m) < \varphi(m)$ is when $2 \| m$ (that is, $m$ is divisible by 2 but not by 4). In this case, the factor of 2 contributes $\varphi(2) = 1$ to the totient but 0 to $\psi$. 

\end{remark}

\subsection{Companion Matrices}

Companion matrices are the fundamental tool for constructing matrices with prescribed characteristic polynomials. They play a central role in the backward direction of our proof.

\subsubsection{Definition and Basic Properties}

\begin{definition}
\label{def:companion}
Given a monic polynomial $p(X) = X^n + a_{n-1}X^{n-1} + \cdots + a_1 X + a_0$ of degree $n \geq 1$, its \emph{companion matrix} $C(p)$ is the $n \times n$ matrix 

\[

C(p) = \begin{pmatrix}
0 & 0 & \cdots & 0 & -a_0 \\
1 & 0 & \cdots & 0 & -a_1 \\
0 & 1 & \cdots & 0 & -a_2 \\
\vdots & \vdots & \ddots & \vdots & \vdots \\
0 & 0 & \cdots & 1 & -a_{n-1}
\end{pmatrix}.

\]

\end{definition}

\begin{theorem}
\label{thm:companion-charpoly}
The characteristic polynomial of $C(p)$ equals $p$. That is, $\det(XI - C(p)) = p(X)$. 

\end{theorem}

\begin{proof}
This is a standard result; we expand $\det(XI - C(p))$ by cofactors along the first row and use induction on the degree. 

\end{proof}

\begin{corollary}
\label{cor:companion-minimal}
If $p$ is irreducible over $\Q$, then $p$ is also the minimal polynomial of $C(p)$. 

\end{corollary}

\begin{proof}
The minimal polynomial divides the characteristic polynomial. If $p$ is irreducible, the only monic divisors of $p$ are 1 and $p$ itself. Since $C(p) \neq I$ for $\deg(p) \geq 1$, the minimal polynomial cannot be 1, so it must be $p$. 

\end{proof}

By the Cayley-Hamilton theorem, every matrix satisfies its characteristic polynomial. For companion matrices, this means the defining polynomial evaluates to zero at the companion matrix: $p(C(p)) = 0$.

\begin{lemma}
\label{lem:companion-divides}
If $p(X)$ divides $q(X)$ in $\Q[X]$, then $q(C(p)) = 0$. 

\end{lemma}

\begin{proof}
Write $q = p \cdot r$ for some polynomial $r$. Then $q(C(p)) = p(C(p)) \cdot r(C(p)) = 0 \cdot r(C(p)) = 0$. 

\end{proof}

\begin{theorem}
\label{thm:companion-power}
If $p(X)$ divides $X^m - 1$, then $C(p)^m = I$. 

\end{theorem}

\begin{proof}
By Lemma~\ref{lem:companion-divides}, $(X^m - 1)(C(p)) = C(p)^m - I = 0$. 

\end{proof}

\subsubsection{Cyclotomic Companion Matrices}

When we apply the companion matrix construction to cyclotomic polynomials, we obtain integer matrices with precisely controlled finite orders.

Recall that the $m$-th cyclotomic polynomial $\Phi_m(X)$ is the minimal polynomial over $\Q$ of a primitive $m$-th root of unity. It has three crucial properties: 

\begin{enumerate}
\item $\Phi_m(X)$ has integer coefficients. 

\item $\Phi_m(X)$ has degree $\varphi(m)$. 

\item $\Phi_m(X)$ divides $X^m - 1$ but does not divide $X^k - 1$ for any $0 < k < m$. 

\end{enumerate}

\begin{theorem}
\label{thm:cyclotomic-order}
For $m \geq 2$, the companion matrix $C(\Phi_m)$ has order exactly $m$. 

\end{theorem}

\begin{proof}
\emph{Upper bound:} Since $\Phi_m \mid X^m - 1$, Theorem~\ref{thm:companion-power} gives $C(\Phi_m)^m = I$.

\emph{Lower bound:} Suppose $C(\Phi_m)^d = I$ for some $0 < d < m$. Then $C(\Phi_m)$ is a root of $X^d - 1$ (as a polynomial in matrices). The minimal polynomial of $C(\Phi_m)$ is $\Phi_m$ (by Corollary~\ref{cor:companion-minimal}, since $\Phi_m$ is irreducible over $\Q$). Therefore $\Phi_m \mid X^d - 1$ in $\Q[X]$.

But the roots of $\Phi_m$ are the primitive $m$-th roots of unity $\zeta$ with $\zeta^m = 1$ and $\zeta^k \neq 1$ for $0 < k < m$. Such a $\zeta$ cannot be a root of $X^d - 1$ when $d < m$. This contradicts $\Phi_m \mid X^d - 1$.

Therefore $\ord(C(\Phi_m)) = m$. 

\end{proof}

\begin{corollary}
\label{cor:cyclotomic-in-orders}
For $m \geq 2$, the companion matrix $C(\Phi_m)$ belongs to $\GL(\varphi(m), \Z)$ and has order $m$. In particular, order $m$ is achievable in dimension $\varphi(m)$. 

\end{corollary}

\begin{proof}
Since $\Phi_m$ has integer coefficients, $C(\Phi_m)$ has integer entries. The matrix has finite order $m$, so it is invertible with $C(\Phi_m)^{-1} = C(\Phi_m)^{m-1}$, which also has integer entries. 

\end{proof}

\begin{example}
The cyclotomic polynomial $\Phi_3(X) = X^2 + X + 1$ has companion matrix 

\[

C(\Phi_3) = \begin{pmatrix} 0 & -1 \\ 1 & -1 \end{pmatrix}.

\]

One can verify directly: $C(\Phi_3)^2 = \begin{psmallmatrix} -1 & 1 \\ -1 & 0 \end{psmallmatrix}$ and $C(\Phi_3)^3 = I$. This $2 \times 2$ integer matrix has order exactly 3. 

\end{example}

\begin{example}
The cyclotomic polynomial $\Phi_6(X) = X^2 - X + 1$ has companion matrix 

\[

C(\Phi_6) = \begin{pmatrix} 0 & -1 \\ 1 & 1 \end{pmatrix}.

\]

This $2 \times 2$ matrix has order 6, demonstrating that order 6 is achievable in dimension 2. 

\end{example}

\subsection{Block Diagonal Matrices}

To construct matrices of arbitrary orders from companion matrices of cyclotomic polynomials, we use block diagonal combinations.

\subsubsection{Definition}

\begin{definition}
\label{def:block-diag}
For matrices $A \in \mathrm{Mat}_{n_1}(\Z)$ and $B \in \mathrm{Mat}_{n_2}(\Z)$, the \emph{block diagonal} matrix is 

\[

\diag(A, B) = \begin{pmatrix} A & 0 \\ 0 & B \end{pmatrix}
\in \mathrm{Mat}_{n_1 + n_2}(\Z).

\]

This extends naturally to any finite number of blocks. 

\end{definition}

\subsubsection{Order of Block Diagonal Matrices}

\begin{lemma}
\label{lem:block-diag-power}
For any matrices $A$ and $B$ and any positive integer $k$, 

\[

\diag(A, B)^k = \diag(A^k, B^k).

\]

\end{lemma}

\begin{proof}
By induction on $k$. The base case $k = 1$ is immediate. For the inductive step: 

\[

\diag(A, B)^{k+1} = \diag(A, B)^k \cdot \diag(A, B)
= \diag(A^k, B^k) \cdot \diag(A, B) = \diag(A^{k+1}, B^{k+1}). \qedhere

\]

\end{proof}

\begin{theorem}
\label{thm:block-diag-order}
If $A$ has finite order $a$ and $B$ has finite order $b$, then $\diag(A,B)$ has order $\lcm(a, b)$. 

\end{theorem}

\begin{proof}
By Lemma~\ref{lem:block-diag-power}, $\diag(A,B)^k = I$ if and only if $A^k = I$ and $B^k = I$. This happens precisely when $a \mid k$ and $b \mid k$, that is, when $k$ is a common multiple of $a$ and $b$. The smallest such positive $k$ is $\lcm(a, b)$. 

\end{proof}

\begin{corollary}
\label{cor:coprime-product}
If $A$ has order $a$, $B$ has order $b$, and $\gcd(a, b) = 1$, then $\diag(A, B)$ has order $ab$. 

\end{corollary}

\begin{proof}
When $\gcd(a, b) = 1$, we have $\lcm(a, b) = ab$. 

\end{proof}

\begin{corollary}
\label{cor:combine-orders}
If order $a$ is achievable in dimension $n_1$ and order $b$ is achievable in dimension $n_2$, then order $\lcm(a, b)$ is achievable in dimension $n_1 + n_2$. 

\end{corollary}

\subsection{The Forward Direction}

We now prove that if an integer matrix has order $m$, then its dimension must be at least $\psi(m)$.

\subsubsection{Minimal Polynomials of Matrices with Finite Order}

Let $A \in \GL(N, \Z)$ be a matrix with finite order $m$. Since $A^m = I$, the polynomial $X^m - 1$ annihilates $A$. The minimal polynomial $\mu_A$ of $A$ (over $\Q$) therefore divides $X^m - 1$.

\begin{lemma}
\label{lem:cyclotomic-factorization}
Over $\Q$, we have the factorization 

\[

X^m - 1 = \prod_{d \mid m} \Phi_d(X),

\]

where the product is over all positive divisors $d$ of $m$. 

\end{lemma}

This is a standard result: every $m$-th root of unity is a primitive $d$-th root of unity for exactly one divisor $d$ of $m$.

\begin{lemma}
\label{lem:minimal-poly-form}
The minimal polynomial $\mu_A$ has the form 

\[

\mu_A = \prod_{d \in S} \Phi_d

\]

for some non-empty subset $S$ of divisors of $m$. Moreover, $\mu_A$ has integer coefficients. 

\end{lemma}

\begin{proof}
Since $\mu_A \mid X^m - 1$ and the cyclotomic polynomials are the irreducible factors of $X^m - 1$ over $\Q$, the minimal polynomial must be a product of distinct cyclotomic polynomials. Since $A \neq 0$, we have $\mu_A \neq 1$, so $S$ is non-empty. The product of cyclotomic polynomials has integer coefficients since each $\Phi_d$ does. 

\end{proof}

\subsubsection{The Order Constraint}

\begin{lemma}
\label{lem:order-lcm}
If $\mu_A = \prod_{d \in S} \Phi_d$, then $\ord(A) = \lcm(S)$. 

\end{lemma}

\begin{proof}
Let $\ell = \lcm(S)$. For each $d \in S$, we have $d \mid \ell$, so $\Phi_d \mid X^\ell - 1$. Therefore $\mu_A \mid X^\ell - 1$, which gives $A^\ell = I$. Thus $\ord(A) \mid \ell$.

Conversely, let $k = \ord(A)$. Then $A^k = I$, so $\mu_A \mid X^k - 1$. For each $d \in S$, the polynomial $\Phi_d$ divides $\mu_A$, hence divides $X^k - 1$. The roots of $\Phi_d$ are primitive $d$-th roots of unity, which are $k$-th roots of unity only if $d \mid k$. Therefore $d \mid k$ for all $d \in S$, giving $\ell = \lcm(S) \mid k$.

Together, $\ord(A) = \ell = \lcm(S)$. 

\end{proof}

\begin{corollary}
\label{cor:lcm-equals-m}
If $\ord(A) = m$, then $\lcm(S) = m$ for the set $S$ such that $\mu_A = \prod_{d \in S} \Phi_d$. 

\end{corollary}

\subsubsection{The Key Inequality}

The heart of the forward direction is the following combinatorial lemma about subsets of divisors.

\begin{lemma}
\label{lem:totient-sum}
Let $m \geq 1$ and let $S$ be a non-empty set of positive divisors of $m$ with $\lcm(S) = m$. Then 

\[

\sum_{d \in S} \varphi(d) \geq \psi(m).

\]

\end{lemma}

\begin{proof}
We use strong induction on $m$.

\emph{Base cases:} For $m = 1$, the only possibility is $S = \{1\}$, and $\varphi(1) = 1 \geq 0 = \psi(1)$. For $m = 2$, we need $S = \{2\}$ or $S = \{1, 2\}$ (since $\lcm(S) = 2$). Either way, $\sum_{d \in S} \varphi(d) \geq \varphi(2) = 1 > 0 = \psi(2)$.

\emph{Inductive step:} Assume $m \geq 3$ and the result holds for all $m' < m$. Write $m = \prod_{p} p^{k_p}$ with $k_p \geq 1$ for finitely many primes $p$.

If $m$ is a prime power, say $m = p^k$ with $k \geq 1$, then $\lcm(S) = p^k$ requires $p^k \in S$. So $\sum_{d \in S} \varphi(d) \geq \varphi(p^k) = \psi_{\mathrm{pp}}(p, k) = \psi(m)$ (except if $(p, k) = (2, 1)$, but then $m = 2$ is a base case).

If $m$ is not a prime power, write $m = ab$ with $\gcd(a, b) = 1$, $a, b > 1$. For each $d \in S$, write $d = d_a d_b$ with $d_a \mid a$ and $d_b \mid b$. Define $S_a = \{d_a : d \in S\}$ and $S_b = \{d_b : d \in S\}$. Then $\lcm(S_a) = a$ and $\lcm(S_b) = b$ (since $\lcm(S) = m = ab$ and the prime factorizations are disjoint).

By the Chinese Remainder Theorem structure: 

\[

\sum_{d \in S} \varphi(d) = \sum_{d \in S} \varphi(d_a) \varphi(d_b)
\geq \sum_{d_a \in S_a} \varphi(d_a) + \sum_{d_b \in S_b} \varphi(d_b) - \varphi(1).

\]

(The inequality uses that the map $d \mapsto (d_a, d_b)$ is injective, and some elementary bounds.)

By the inductive hypothesis, $\sum_{d_a \in S_a} \varphi(d_a) \geq \psi(a)$ and $\sum_{d_b \in S_b} \varphi(d_b) \geq \psi(b)$. Since $\psi(m) = \psi(a) + \psi(b)$ by coprime additivity, we obtain $\sum_{d \in S} \varphi(d) \geq \psi(m)$. 

\end{proof}

\subsubsection{Proof of the Forward Direction}

\begin{theorem}
\label{thm:forward}
If $A \in \GL(N, \Z)$ has order $m$, then $\psi(m) \leq N$. 

\end{theorem}

\begin{proof}
Let $\mu_A = \prod_{d \in S} \Phi_d$ be the minimal polynomial of $A$. By Corollary~\ref{cor:lcm-equals-m}, $\lcm(S) = m$.

The degree of $\mu_A$ is $\sum_{d \in S} \deg(\Phi_d) = \sum_{d \in S} \varphi(d)$.

By Lemma~\ref{lem:totient-sum}, $\sum_{d \in S} \varphi(d) \geq \psi(m)$.

The minimal polynomial divides the characteristic polynomial, so $\deg(\mu_A) \leq \deg(\chi_A) = N$.

Combining: $\psi(m) \leq \sum_{d \in S} \varphi(d) = \deg(\mu_A) \leq N$. 

\end{proof}

\subsection{The Backward Direction}

We now show that every order satisfying the necessary condition is actually achievable.

\subsubsection{Prime Power Orders}

By Corollary~\ref{cor:cyclotomic-in-orders}, for any prime power $p^k$ with $k \geq 1$, the companion matrix $C(\Phi_{p^k})$ achieves order $p^k$ in dimension $\varphi(p^k)$.

\begin{lemma}
\label{lem:prime-power-achievable}
For a prime $p$ and $k \geq 1$ with $(p, k) \neq (2, 1)$, order $p^k$ is achievable in dimension $\psi_{\mathrm{pp}}(p, k) = \varphi(p^k)$. 

\end{lemma}

\begin{proof}
Take $C(\Phi_{p^k})$, which has dimension $\varphi(p^k)$ and order $p^k$. 

\end{proof}

\subsubsection{Order 2 is Free}

The special case $(p, k) = (2, 1)$ requires separate treatment.

\begin{lemma}
\label{lem:order-two-free}
For any $N \geq 1$, the $N \times N$ matrix $-I$ has order 2. 

\end{lemma}

\begin{proof}
We have $(-I)^2 = I$ and $-I \neq I$ (since $N \geq 1$). 

\end{proof}

This is why $\psi(2) = 0$: achieving order 2 costs nothing beyond having at least one dimension.

\begin{lemma}
\label{lem:double-odd}
If $A \in \GL(N, \Z)$ has odd order $k$, then $-A$ has order $2k$. 

\end{lemma}

\begin{proof}
We have $(-A)^{2k} = (-1)^{2k} A^{2k} = A^{2k} = (A^k)^2 = I$.

For the lower bound, suppose $(-A)^j = I$ for some $j \geq 1$. Then $(-1)^j A^j = I$, so $A^j = (-1)^j I$. If $j$ is odd, then $A^j = -I$, so $A^{2j} = I$, giving $k \mid 2j$. Since $k$ is odd, $k \mid j$. But then $A^j = A^k = I \neq -I$, contradiction. So $j$ is even, say $j = 2\ell$, and $A^{2\ell} = I$, giving $k \mid 2\ell$. Since $k$ is odd, $k \mid \ell$, so $j = 2\ell \geq 2k$. Thus $\ord(-A) = 2k$. 

\end{proof}

\begin{corollary}
\label{cor:twice-odd}
If order $k$ (odd) is achievable in dimension $N$, then order $2k$ is achievable in dimension $N$. 

\end{corollary}

\subsubsection{General Construction}

\begin{theorem}
\label{thm:backward}
If $\psi(m) \leq N$, then some matrix in $\GL(N, \Z)$ has order $m$. 

\end{theorem}

\begin{proof}
We construct a matrix of order $m$ in dimension $\psi(m)$, then pad with identity blocks to reach dimension $N$ if needed.

Write $m = 2^a \cdot m'$ where $m'$ is odd and $a \geq 0$.

\emph{Case 1: $a = 0$ (m is odd).} Write $m' = \prod_{i} p_i^{k_i}$ with distinct odd primes $p_i$ and $k_i \geq 1$. For each $i$, let $A_i = C(\Phi_{p_i^{k_i}})$, which has order $p_i^{k_i}$ and dimension $\varphi(p_i^{k_i})$.

Form $A = \diag(A_1, A_2, \ldots)$. By Theorem~\ref{thm:block-diag-order} and induction, $A$ has order $\lcm(p_1^{k_1}, p_2^{k_2}, \ldots) = m'$ (since the prime powers are pairwise coprime). The dimension is $\sum_i \varphi(p_i^{k_i}) = \psi(m')$.

\emph{Case 2: $a = 1$ (m = 2m' with m' odd).} Construct $A$ of order $m'$ in dimension $\psi(m')$ as above. Then $-A$ has order $2m'$ by Lemma~\ref{lem:double-odd}, still in dimension $\psi(m') = \psi(2m') = \psi(m)$.

\emph{Case 3: $a \geq 2$.} Include a block $C(\Phi_{2^a})$ of order $2^a$ and dimension $\varphi(2^a) = 2^{a-1}$. Combine with blocks for the odd part $m'$ as in Case 1. The total order is $\lcm(2^a, m') = 2^a m' = m$, and total dimension is $\varphi(2^a) + \psi(m') = \psi_{\mathrm{pp}}(2, a) + \psi(m') = \psi(m)$.

In all cases, we achieve order $m$ in dimension exactly $\psi(m)$.

If $\psi(m) < N$, adjoin identity blocks $I_{N - \psi(m)}$: $\diag(A, I_{N-\psi(m)})$ still has order $m$ (since $\lcm(m, 1) = m$) and has dimension $N$. 

\end{proof}

\begin{example}
We have $12 = 4 \cdot 3 = 2^2 \cdot 3$, so we are in Case 3 with $a = 2$ and $m' = 3$.

For the factor $2^2 = 4$: $C(\Phi_4) = C(X^2 + 1) = \begin{psmallmatrix} 0 & -1 \\ 1 & 0 \end{psmallmatrix}$, which has order 4 and dimension $\varphi(4) = 2$.

For the factor 3: $C(\Phi_3) = \begin{psmallmatrix} 0 & -1 \\ 1 & -1 \end{psmallmatrix}$, which has order 3 and dimension $\varphi(3) = 2$.

The block diagonal 

\[

\diag(C(\Phi_4), C(\Phi_3)) = \begin{pmatrix}
0 & -1 & 0 & 0 \\
1 & 0 & 0 & 0 \\
0 & 0 & 0 & -1 \\
0 & 0 & 1 & -1
\end{pmatrix}

\]

has order $\lcm(4, 3) = 12$ and dimension $2 + 2 = 4 = \psi(12)$. 

\end{example}

\subsection{The Crystallographic Restriction Theorem}

We now combine the forward and backward directions.

\subsubsection{The Main Result}

\begin{theorem}
\label{thm:main}
Let $N \geq 1$ and $m \geq 1$. An $N \times N$ integer matrix can have finite order $m$ if and only if $\psi(m) \leq N$. 

\end{theorem}

\begin{proof}
\emph{($\Rightarrow$)} This is Theorem~\ref{thm:forward}.

\emph{($\Leftarrow$)} This is Theorem~\ref{thm:backward}. 

\end{proof}

\begin{definition}
For $N \geq 1$, let $\mathrm{Ord}_N$ denote the set of all positive integers $m$ such that some matrix in $\GL(N, \Z)$ has order $m$. 

\end{definition}

\begin{corollary}
$\mathrm{Ord}_N = \{m \geq 1 : \psi(m) \leq N\}$. 

\end{corollary}

\subsubsection{The Classical Crystallographic Restriction}

\begin{corollary}
\label{cor:classical}
In dimensions 2 and 3, the achievable orders are exactly $\{1, 2, 3, 4, 6\}$. 

\end{corollary}

\begin{proof}
From Lemma~\ref{lem:psi-values}, $\psi(m) \leq 2$ if and only if $m \in \{1, 2, 3, 4, 6\}$: 

\begin{itemize}
\item $\psi(1) = \psi(2) = 0 \leq 2$ \checkmark 

\item $\psi(3) = \psi(4) = \psi(6) = 2 \leq 2$ \checkmark 

\item $\psi(5) = 4 > 2$ $\times$ 

\item $\psi(7) = 6 > 2$ $\times$ 

\end{itemize}

Since $\psi(m) \leq 3$ if and only if $\psi(m) \leq 2$ (as $\psi(m)$ is always even for $m > 2$), the achievable orders in dimensions 2 and 3 coincide. 

\end{proof}

This explains why crystals in our three-dimensional world exhibit only 2-, 3-, 4-, and 6-fold rotational symmetries: these correspond to the elements of orders 2, 3, 4, and 6 in $\GL(3, \Z)$.

\subsubsection{Dimension Parity}

\begin{corollary}
\label{cor:parity}
For $k \geq 1$, the sets $\mathrm{Ord}_{2k}$ and $\mathrm{Ord}_{2k+1}$ are equal. 

\end{corollary}

\begin{proof}
For $m > 2$, the value $\psi(m) = \sum_{(p,k) \neq (2,1)} \varphi(p^k)$ is a sum of terms $\varphi(p^k) = (p-1)p^{k-1}$. Each such term is even: if $p$ is odd, then $p - 1$ is even; if $p = 2$ and $k \geq 2$, then $p^{k-1} = 2^{k-1}$ is even.

Thus $\psi(m)$ is even for all $m \geq 1$. The condition $\psi(m) \leq 2k$ is equivalent to $\psi(m) \leq 2k + 1$. 

\end{proof}

This means that passing from an even dimension to the next odd dimension never unlocks new achievable orders.

\subsubsection{First Occurrence Table}

The following table shows when each small order first becomes achievable:

\begin{center}...\end{center}We observe: 

\begin{itemize}
\item Orders 1 and 2 are achievable in dimension 1 (the smallest possible). 

\item Orders 3, 4, 6 first appear in dimension 2. 

\item Orders 5, 8, 10, 12 first appear in dimension 4. 

\item Orders 7, 9 first appear in dimension 6. 

\item Order 11 requires dimension 10. 

\end{itemize}

\begin{thebibliography}...\end{thebibliography}
\end{document}
